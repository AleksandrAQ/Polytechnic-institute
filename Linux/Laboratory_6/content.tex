\setcounter{page}{3}
\setlength{\parskip}{5pt}
\setlength{\parindent}{1,25cm}
\linespread{1.3}
\pagestyle{plain}
\fontsize{14pt}{16.8pt}\selectfont

\newpage
\begin{center}
\anonsection{Введение}
\label{sec:intro}
\end{center}
\par
\fontsize{14pt}{16.8pt}\selectfont
Цель курсовой работы - изучение набор макрорасширений системы компьютерной вёрстки документов TeX: LaTeX. В рамках работы необходимо изучить приниципы и способы создания текстовых документов ( в т.ч. математических формул), посредством встроенного инструментария языка LaTeX.
\par
\fontsize{14pt}{16.8pt}\selectfont
Задача работы - создание документа формата *.pdf, содержащего оформленный в рамках методической инструкции[1] описание упражнений дисциплины <<Алгритмизация и основы программирования>>.

\newpage
\begin{center}
\section{\fontsize{16pt}{16.8pt}Описание предметной области}
\end{center}
\par
\fontsize{14pt}{16.8pt}\selectfont
LaTeX — наиболее популярный набор макрорасширений системы компьютерной вёрстки TeX, который облегчает набор сложных документов.[2]
\par
\fontsize{14pt}{16.8pt}\selectfont
Пакет позволяет автоматизировать многие задачи набора текста и подготовки статей, включая набор текста на нескольких языках, нумерацию разделов и формул, перекрёстные ссылки, размещение иллюстраций и таблиц на странице, ведение библиографии и др. Кроме базового набора существует множество пакетов расширения LaTeX. Первая версия была выпущена Лесли Лэмпортом в 1984 году; текущая версия, LaTeX2e, после создания в 1994 году испытывала некоторый период нестабильности, окончившийся к концу 1990-х годов, а в настоящее время стабилизировалась.
\par
\fontsize{14pt}{16.8pt}\selectfont
Общий внешний вид документа в LaTeX определяется стилевым файлом. Существует несколько стандартных стилевых файлов для статей, книг, писем и т. д., кроме того, многие издательства и журналы предоставляют свои собственные стилевые файлы, что позволяет быстро оформить публикацию, соответствующую стандартам издания.
\par
\fontsize{14pt}{16.8pt}\selectfont
Термин LaTeX относится только к языку разметки, он не является текстовым редактором. Для того, чтобы создать документ с его помощью, надо набрать tex-файл с помощью какого-нибудь текстового редактора. В принципе, подойдёт любой редактор, но большая часть людей предпочитает использовать специализированные, которые так или иначе облегчают работу по набору текста LaTeX-разметки.
\par
\fontsize{14pt}{16.8pt}\selectfont
Будучи распространяемым под лицензией LaTeX Project Public License, LaTeX относится к свободному программному обеспечению.
\par
\fontsize{14pt}{16.8pt}\selectfont
Сравним используемый повсеместно подход к оформлению документов с подходом, основанном на языке TeX.
\par
\fontsize{14pt}{16.8pt}\selectfont
Современные текстовые процессоры используют технологию WISIWYG, что предполагает акцентирование внимания на том, как документ будет выглядеть на печати. LaTeX является программой языка разметки, нацеленной на оформление документов. Особенностью LaTeX является разделение правил создания содержания документа и правил его оформления. Фактически, внимание пользователя фокусируется на содержании документа и его логической структуре, а всю работу по верстке документа берет на себя компьютер, основываясь на выбранном классе документа. Кроме того, автоматизируется не только размещение иллюстраций, работа с таблицами, нумерация формул, но и работа с перекрестными ссылками, библиографией и др.

\newpage
\begin{center}
\section{\fontsize{16pt}{16.8pt}Представление сборника задач в формате TeX }
\end{center}
\par
\fontsize{14pt}{16.8pt}\selectfont
В рамках курсовой работы была поставлена задача представления текста упражнений по предмету <<Алгоритмизация и основы программирования>> в формате TeX. Необходимо было оформить задания в формате, установленном в методическом указании[1].
\par
\fontsize{14pt}{16.8pt}\selectfont
Оформленные упражнения по предмету <<Алгоритмизация и основы программирования>>, представленны в соотвествующих подразделах.
\par

\subsection{\fontsize{14pt}{16.8pt}Упражнение 1.7}
\noindent
Для значения переменных $a$ и $b$. Поменять местами эти значения, расширив задачу двумя способами а) с использованием дополнительной переменной б) без использования дополнительной переменной, но используя операции сложения и вычетания значений заданынх переменных.

\subsection{\fontsize{14pt}{16.8pt}Упражнение 2.3}
\noindent
Даны значения переменных $a$, $b$, и $c$. Напечатать значения корней, их количество и тип ( вещественные, комплексные или мнимые).

\subsection{\fontsize{14pt}{16.8pt}Упражнение 3.1б}
\noindent
Для заданного массива $A$ из ста элементов вещественного типа вычислить сумму квадратов его элементов, начиная с первого.
  
\subsection{\fontsize{14pt}{16.8pt}Упражнение 4.1б}
\noindent
Протабулировать функции одной переменной:\\[2mm]
    $ f(x) = \ln x  +\sin^2 x,$\\
    $0.25 \leq x \leq 0.75,\triangle x = 0.05 $

\subsection{\fontsize{14pt}{16.8pt}Упражнение 5.3}
\noindent
В последовательности из 1000 чисел найти среднее арифмитическое отрицательное число и их количество.

\subsection{\fontsize{14pt}{16.8pt}Упражнение 6.1б}
\noindent
Вычислить с заданной абсолютной погрежностью $ABSERR$ значения элементарных функций при заданном значении ар гумента $x$::\\[2mm]
    $$ cos x = 1 - \frac{x^2}{2!} + \frac{x^4}{4!} - \frac{x^6}{6!} + ...$$
    
\subsection{\fontsize{14pt}{16.8pt}Упражнение 7.4}
\noindent
В массиве $A$(10) с элементами, упорядоченными по возрастанию, исключить пятый элемент, добавить вместо него заданное значение и оставляя массив упорядоченным.

\subsection{\fontsize{14pt}{16.8pt}Упражнение 7.7}
\noindent
Вычислить значение величины по заданным значениям трехмерной матрицы: $$q = \sum_{i=1}^{n} \prod_{j=1}^{m} \sum_{k=1}^{l}$$
\subsection{\fontsize{14pt}{16.8pt}Упражнение 7.17б}
\noindent
В массиве $A(100,50)$ найти элемент, являющийся наименьшим.    
\subsection{\fontsize{14pt}{16.8pt}Упражнение 7.20}
\noindent
Сформировать одномерный массив $B$ из элементов двумерного массива $A$(100,20), сумма индексов которых четная (нечетная).

\subsection{\fontsize{14pt}{16.8pt}Упражнение 7.43}
\noindent
Дана матрица $A$(30,30). Просматривая главную диагональ сверху вниз, а затем побочную диагоаль снизу вверх, выбрать отрицательные элементы и сформировать из низ отдельный массив.

\subsection{\fontsize{14pt}{16.8pt}Упражнение 8.2}
\noindent
Составить процедуру умножения матрицы $B(M, N)$ на вектор $C(N)$. Применить ее для заданного вектора $A(10)$ и матрицы $D(15,10)$, элементы которой нао предварительно сформировать по правилу $d_{ij}=i(j+5)$.

\newpage
\begin{center}
\anonsection{\fontsize{16pt}{16.8pt}Заключение}
\label{sec:close}
\end{center}
\par
\fontsize{14pt}{16.8pt}\selectfont
В рамках курсовой работы были оформлены упражнения по дисциплине <<Алгоритмизация и основы програмирования>> в заданном формате[1].
В процессе выполнения работы был изучен синтаксис и особенности языка разметки Latex. 
\par
\fontsize{14pt}{16.8pt}\selectfont
В процессе использования подхода оформления документа с помощью языка разметки LaTeX, были выделены следующие преимущества:
\begin{spacing}{0.9}
\fontsize{14pt}{16.8pt}\selectfont
\begin{enumerate} 
  \item Легкая смена оформления документа.
  \item Переносимость результата между устройствами.
 \itemБыстрый способ набора математических формул.
\itemЛегкая смена оформления документа.
\itemПростая нумерация формул.
\itemРабота с библиографией.
\itemПлавающие объекты.
\itemПоддержка макросов.
\itemПростота генерации документов из других программ.
\itemБольшое количество дополнительных пакетов.
\end{enumerate}
\end{spacing}

\newpage
\begin{center}
\anonsection{\fontsize{16pt}{16.8pt}Список использованных источников}
\end{center}
\par
1. Регламент содержания, оформления, организации выполнения и защиты курсовых проектов и курсовых работ / утвержден учебно-методическим советом / СПБПУ – [СПБ.,2018]. – 24 с.
\par
2. С.М. Львовский, LaTeX: подробное описание // Предварительная рабочая версия. / СПбГУ – [СПБ., 2017]. – URL: geo.phys.spbu.ru/LDUS/ LaTeX-Lvovsky.pdf
– (дата обращения: 02.06.2019).
